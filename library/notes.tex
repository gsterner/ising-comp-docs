\documentclass[11pt]{article}

% Margins
\topmargin=-0.45in
\evensidemargin=0in
\oddsidemargin=0in
\textwidth=6.5in
\textheight=9.0in
\headsep=0.25in

\title{Raw notes on Ising Computation}
\author{Gustaf Lundborg}
\date{\today}

\begin{document}
\maketitle
\pagebreak

% Optional TOC
% \tableofcontents
% \pagebreak

%--Paper--

\section{2021-06-03}
\begin{itemize}
\item How to create Ising circuits?
\item What are the scales we are working on?
\item Connect several using systems as Ising spins
\item E.g. ising chains that are separate, but can be connected at different times. There are many ways of creating dynamics
\item What can a small Ising system calculate? Interesting to see it as a representation of a binary number. But only next door neighbor interactions of bits are not that interesting. Should probably have full connectivity. E.g. four spins without self connectivity.
\item Would it be possible to make a random number generator from an Ising circuit? IE. from a critical state (needs investigating).
\item If you randomize an Ising chain it should be possible to get a random sequence of up and down, which is perfect for Brownian motion.
\item Could you get this through critical stat? Is critical state needed. Or can I just do a simulation to a "ground state"?
\item will there be a critical state in a fully connected Ising chain?
\end{itemize}


%% \pagebreak
%% \section{Section 2}

%--/Paper--

\end{document}
